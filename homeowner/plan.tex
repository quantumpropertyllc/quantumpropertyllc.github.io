\documentclass[11pt]{article}
\usepackage{array, latexsym, graphicx}
\usepackage{times}
\usepackage{hyperref}
\usepackage{amsfonts}
\urlstyle{same}
\textwidth 15 cm
\textheight 22 cm
\voffset=-0.5 cm
\hoffset=-1.5 cm
\begin{document}

\title{Retirement Plan}
\author{Yongge Wang\\
UNC Charlotte, USA\\
yonwang@uncc.edu}
\author{}
\date{}

%\maketitle

\noindent
{\bf 401(k) and 457(b)}
\begin{table}[htp]
\begin{center}
\begin{tabular}{|c|c|c|} \hline
{\bf Date} & {\bf 401(k)} & {\bf 457(b)}\\ \hline
02/08/2010 & \$114,131.61 & \$19,298.19\\ \hline
01/21/2011 & \$142,920.32 & \$39,247.66\\ \hline
11/14/2013 & \$224,349.59 & \$105,118.37\\ \hline
12/11/2017 & \$351,936.36& \$208,528.02\\ \hline
7/23/2018 & \$382,171.81& \$231,870.07\\ \hline
2/15/2020 & \$467,050.84 & \$301,199.65\\ \hline
5/10/2021 & \$584,099.59 & \$399,789.80\\ \hline
4/20/2022 & \$609,311.51& \$407,790.89\\ \hline
2/14/2023 & \$616,154.81 & \$418,927.61 \\ \hline
1/19/2024 & \$693,955.66& \$472,312.91 \\ \hline
Account info & yonwang2002 & *1 \\ \hline
\end{tabular}
\caption{page: \url{https://ncplans.retirepru.com}}
\label{401k}
\end{center}
\end{table}

\noindent
{\bf UNCC Retirement Plan}
\begin{table}[htp]
\begin{center}
\begin{tabular}{|c|c|c|c|} \hline
{\bf Date} & Balance & Employer & Employee\\ \hline
2/08/2010&\$104,333.68& \$50,357.52 & \$44,116.82\\ \hline
1/21/2011&\$131.717.62&\$56,948.86 & \$49,898.79\\ \hline
11/14/2013&\$209,562.21&\$77,585.28 & \$68,000.92\\ \hline
12/11/2017&\$350,379.81&\$109,844.12 & \$96,297.90\\ \hline
7/23/2018&\$377,818.46&\$117,072.96& \$102,639.03\\ \hline
2/15/2020&\$448,313.85&\$238,662.55 & \$209,216.35\\ \hline
5/10/2021&\$552,393.93&\$293,967.83 & \$257,704.26\\ \hline
4/20/2022&\$575,845.91&\$268,543.09 & \$306,326.67 \\ \hline
2/14/2023&\$582,424.63 &\$309,836.88 & \$271,625.16\\ \hline
1/19/2024 &\$638,292.48&\$339,570.82 & \$297,696.90\\ \hline
Account info & yonwang &Tf*!1&\\ \hline
\end{tabular}
\caption{page: \url{http://www.tiaa-cref.org/}}
\label{tiaa}
\end{center}
\end{table}


\noindent
{\bf Life Insurance}
\begin{table}[htp]
\begin{center}
\begin{tabular}{|c|c|c|c|c|} \hline
{\bf Insurance Name}&{\bf Company}&{\bf Account}&{\bf Amount}&{\bf Be-Summary}\\ \hline
Group Term Life & ING 1-877-464-5111 && \$100,000 & unknown \\ \hline
Cancer & Allstate 904-992-1776 & 86573142 && \\ \hline
AD\&D Insurance& UNUM 800-257-0930 &LHNCV000760&\$200,000& UK\\ \hline
American United Life& 800-336-3085 ext 107&&\$200,000& \\ \hline
\end{tabular}
\caption{Related Insurance}
\label{insurance}
\end{center}
\end{table}



\section{Other bank information}
\begin{itemize}
\item \url{https://www.ncsecu.org/0010191354/*02/}
\item \url{http://home.ingdirect.com/36348554/}
\item \url{http://www.cibc.com/}
\end{itemize}

\clearpage

Let $A$ be a set with $k$ letters and 
$x = x_1x_2x_3…x_n$ be a string with length $n$ over $A$. 
A rare subchain $y$ of $x$ with length $m$ needs to fulfill the following properties
\begin{itemize}
\item $y$ is a substring of $x$
\item we can express $y$ with nonconsecutive letters in $x$ (for example,  ``os'' 
is a rare subchain of ``proposed'' because ``os'' is a substring of ``propOSed'' and we can express ``os'' 
with nonconsecutive letters in ``prOpoSed'')
\item there is no sub string $y'$ of $x$ with a lengh greater than $m$ that can be expressed with nonconsecutive letters in $x$ (``os'' is a rare subchain of ``proposed'' because there is no other substring of ``proposed'' with length greater than 2 that can be expressed by non-consecutive letters)
\end{itemize}
Given a positive integer $r$,  how many strings $x$ over $A$  has a rare subchain of length $r$.

Solution:  Let $y$ be a string over $A^r$ that we want it to be the rare subchain.  It is clear that there are $k^r$ such kind of strings.  If we could 
compute the number  $\alpha$ of strings $x$ such that $y$ is a rare subchain of $x$, then we know that there 
are $\alpha k^r$ strings $x$ over $A$ such that it has a rare subchain of length $r$. To compute the number $\alpha$, we distinguish 
the following cases:
\begin{enumerate}
\item $y$ starts at the first position of $x$, that is, $x=yz$ for some string of $z\in A^{n-r}$: In this case, we only need to 
calculate the number $\beta_1$ of strings that does not have $y$ as a rare subchain.  Then we know that the nubmer of strings 
that has $y$ as a rare sunbchain is $\alpha_1=k^{n-r}-\beta_1$. $\beta_1$ is of the sum of the following two cases:
\begin{enumerate}
\item there is an $x_0\in A$ such that $yx_0$ is a substring of $x$ and 
$yx_0$ can be expressed with nonconsecutive letters in $x$: 
there are $k$ candidates for $x_0$.   For each $x_0$,  this happens if 
$x_0$ appears in one of the positions $r+2, \cdots, n$.  That is, for each $x_0$, we have a total of 
$k^{n-r-1}-(k-1)^{n-r-1}$ such kind of $x$.  Here the number $(k-1)^{n-r-1}$
is the number of strings that do not contain $x_0$ in the positions $r+2, \cdots, n$.
\item $y$ is not expressed as a string of nonconsecutive letters in $x$ and and $yx_0$  is not expressed as
 a string of nonconsecutive letters in $x$ where $yx_0$ is a substring of $x$ (where $x_0$ is not the same
as the last letter of $y$): 
In this case,  there are $k-1$ candidates for $x_0$.  Furthermore, $x_0$ and 
the last letter of $y$ should not appear in the positons of $r+2, \cdots, n$.  In a summary, there 
is a total of $(k-1)(k-2)^{n-r-1}$ such kind of $x$.
In a summary,  there is a total of $(k-1)(k-2)^{n-r-1}$ such kind of $x$
\end{enumerate}
In a summary, we have $\beta_1=k^{n-r-1}-(k-1)^{n-r-1} +(k-1)(k-2)^{n-r-1}$. 
That is, $\alpha_1=k^{n-r}-k^{n-r-1}+(k-1)^{n-r-1}-(k-1)(k-2)^{n-r-1}$.
\item $y$ starts at the second position of $x$, that is, $x=z_1yz$ for some string of $z\in A^{n-r-1}$ and for $z_1\in A$

\item ...
\item $y$ starts at the position $n-r+1$ of $x$, that is, $x=zy$ for some string of $z\in A^{n-r}$: 
\end{enumerate}
Then $\alpha$ is the sum of these these numbers $\alpha+\alpha_1+\cdots+\alpha_{n-r+1}$.



\clearpage
\noindent
for a prime $p$ and an integer $m\in [2,p+1]$,  consider the set $G(p,m)$ of polynomials 
$$f(x)=x+c_2x^2+\cdots+c_mx^m$$
where $c_2,\cdots, c_m\in Z/pZ$.  Define $f*g(x)=f(g(x))$.  Questons:
\begin{enumerate}
\item check that $G(p,m)$ is a group
\item find a representative of each conjugacy class of $G(p,m)$
\item find the number of conjugacy classes of $G(p,m)$
\end{enumerate}

{\bf Answer}:
For $m=2$,  let $f_1(x)=x+c_2x^2$ and $f_2(x)=x+c_2'x^2$. Then $f_1(f_2(x))=x+(c_2+c_2')x^2$.
That is, for each $f(x)=x+c_2x^2$, we have $f^{-1}(x)=x+(p-c_2)x^2$. For any given $f(x)=x+c_2x^2$, $g(x)=x+c_2'x^2$,
and $h(x)=x+c_2''x^2$. Thus 
$$f=h*g*h^{-1}\mbox{ if and only if }c_2=c_2'+p$$
That is,
$$f(x)\equiv g(x)\mbox{ if and only if }c_2=c_2'+p$$
\begin{enumerate}
\item done
\item For $f(x)=x+c_2x^2$, the conjugacy class is $\{x+c_2x^2, x+(p-c_2)x^2\}$
\item There are $\frac{p+1}{2}$ conjugacy classes: $[x],[x+x^2], \cdots, [x+\frac{p-1}{2}x^2]$
\end{enumerate}

For $m=2$.  Let $f(x)=x+c_2x^2+c_3x^3$ and $g(x)=x+c_2'x^2+c_3'x^3$. Then
$f*g(x)=x+(c_2+c_2')x^2+(c_3+c_3'+2c_2c_2')x^3$.  In order for $f=g^{-1}$, we need
$c_2+c_2'=p$ and $c_3+c_3'+2c_2c_2'=p$.  The analysis could continue.

\end{document}




















